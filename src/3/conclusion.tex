\graphicspath{{9/figures/}}
\chapter{Conclusion}
\label{chp:conclusion}

\section{Findings}

This work has investigated the range of George Whitefield's voice and the accuracy of Benjamin Franklin's auditory experiment to find Whitefield's maximum audience size. The investigation has required historical, archaeological, and meteorological research as well as physics-based reasoning and numerical acoustic simulations. While the author does not claim to be an expert in all these diverse fields, with the help and advice of others who are, the evidence discussed here makes a strong case for the trustworthiness of the acoustic models constructed during this research. These models suggest that Whitefield, along with other trained vocalists, could produce average vocal \gls{spl} values of about 90 dB$_A$ at a distance of 1 m. Based on Whitefield's vocal level, it has been simulated that Whitefield could have reached a crowd of up to 50,000 people under ideal acoustic conditions. Even assuming higher noise levels or lower crowd density, the majority of Whitefield's large crowds of 20,000-30,000 seem acoustically reasonable based on the data provided by Franklin's experiment. Since Whitefield's voice is projected to be as loud as any measured voice today, the crowd sizes projected here may also be good maximum values for any human gathering in the pre-amplified era.

Franklin's \gls{mia} estimation is slightly lower but still very close to those generated by the computer models, indicating that his semicircular assumption still provides a good first-order approximation for this quantity without further information about source directivity or environmental contributions. However, Franklin's density value is probably overly optimistic by at least a factor of 2. Thus this work provides a better lens for understanding Franklin's early scientific approach before his more well known work in electromagnetism. 

\section{Implications}

Since the publication of the C.P. Snow essay {\it The Two Cultures} \cite{Snow1959} it has been a common scene for humanists to be nervous about scientists' claims to represent the future of human knowledge. It is possible that much of the resistance to Digital Humanities research stems from an unwillingness to cede intellectual ground to overly confident scientists wielding equations and computers. 

However, properly considered, this should not be an area of conflict because science and the humanities have not only different tools, but also different goals.\footnote{Indeed, Snow himself was concerned that without sufficient scientific understanding, humanists and others would be overly-deferential to scientists in positions of authority \cite{Snow1960}.} Science, while ill-equipped to handle questions of meaning, value, or purpose, is quite well suited to counting, which is the basis of this project and indeed most of physics at a fundamental level. Humanities disciplines like history require empirical facts to interpret, and science provides the best tools for providing these basic facts for further analysis. 

In a similar way, it is not the author's intention to ``run the table'' on epistemological authority on a historical issue. Rather, a historically significant numerical question has been examined from a scientific viewpoint to provide a quantitative answer (or at least to quantify the uncertainty remaining in the answer). It is hoped that trained historians will further apply and interpret the findings from this research in a broader historical context.

\section{Summing Up}

Whitefield declared in 1739 that

\begin{quote}
The Christian world is in a deep sleep. Nothing but a loud voice can waken them out of it! \cite{Vaudry2003}
\end{quote}

\noindent This statement nicely captures Whitefield's lasting significance. Not only did his relentless travel schedule and singleminded devotion to his mission succeed in awakening a religious movement that sparked lasting social, political, and ecclesiastical reform, but he also did so with the loudest of voices - one that (metaphorically speaking, of course\footnote{Acoustic metaphors should always be used sparingly, especially in scientific works.}) continues to resound through the ages.

